\documentclass[a4paper,11pt]{article}
\usepackage[spanish]{babel}
\usepackage{amsmath}
\usepackage[official]{eurosym}

\author{Jos\'e Ignacio Croce Busquets\footnote{This document is licensed under a Creative Commons Attribution 4.0 Unported License.  You should have received a copy of the license along with this work.  If not, see http://creativecommons.org/licenses/by/4.0/.}}
\date{30 de Diciembre de 2009}
\title{C\'alculo de la posici\'on de encuentro de las agujas del reloj}

\pagestyle{plain}

\newcommand{\boxedeqn}[1]{%
  \[\fbox{%
      \addtolength{\linewidth}{-2\fboxsep}%
      \addtolength{\linewidth}{-2\fboxrule}%
      \begin{minipage}{\linewidth}%
      \begin{equation}#1\end{equation}%
      \end{minipage}%
    }\]%
}

\begin{document}
\maketitle
\begin{description}
\item[Problema:]
Para un reloj de agujas anal\'ogico (o sea que las agujas se mueven a
velocidad constante y sin pegar saltitos) y partiendo de las 12hs en punto,
calcular las horas exactas a las que las agujas de los minutos y las de las
horas se vuelven a superponer.

\item[Soluci\'on:]
Dado que la aguja de los minutos da una vuelta completa por hora, podemos
deducir la velocidad angular de la misma como:
\begin{equation}
\omega_m = \frac{360\,^\circ}{60\,min} = 6\,^\circ/min \label{omegam}
\end{equation}

Mientras que para la aguja de las horas tenemos:
\begin{equation*}
\omega_h = \frac{360\,^\circ}{12\,h} = 30\,^\circ/h
\end{equation*}

Que, expresado en grados por minuto resulta en:
\begin{equation}
\omega_h = \frac{30\,^\circ}{1\,h} \times \frac{1\,h}{60min} = 0,5\,^\circ/min
\label{omegah}
\end{equation}

Si tomamos la posici\'on inicial (o sea el punto de $0\,^\circ$) como las 12
horas en punto, tenemos que el \'angulo $\alpha$ que avanza cada una de las
agujas en funci\'on del tiempo transcurrido es:
\begin{align}
\alpha_m(t) &= \omega_mt \label{alpham} \\
\alpha_h(t) &= \omega_ht \label{alphah}
\end{align}

Partiendo de esta posici\'on inicial de las 12 horas, el pr\'oximo instante
(que denominaremos $t_1$) al que las agujas se vuelvan a encontrar ser\'a
cuando la aguja de los minutos haya completado una vuelta completa. O sea:
\begin{equation}
\alpha_m(t_1) = \alpha_h(t_1) + 360\,^\circ
\label{meet1}
\end{equation}

Y, reemplazando \eqref{alpham} y \eqref{alphah} en \eqref{meet1} resulta:
\begin{equation*}
\omega_m t_1 = \omega_h t_1 + 360\,^\circ
\end{equation*}

De donde podemos depejar $t_1$ como:
\begin{gather*}
\omega_m t_1 - \omega_h t_1 = 360\,^\circ\\
t_1 (\omega_m - \omega_h) = 360\,^\circ\\
t_1 = \frac {360\,^\circ} {\omega_m - \omega_h} \\
\end{gather*}

Que, con los valores de velocidad angular deducidos en \eqref{omegam} y
\eqref{omegah} resulta enun valor de $t_1$ igual a:
\boxedeqn{%
t_1 = \frac{360\,^\circ}{6\,^\circ/min - 0,5\,^\circ/min} =
65.\overline{45}min =
1\,\text{hora}\,5\,\text{min}\,27.\overline{27}\,\text{seg}
\nonumber
}

Para el pr\'oximo punto de encuentro (que denominaremos $t_2$), la aguja de
los minutos habr\'a dado ya dos vueltas completas. Y para el tercer punto de
encuentro las vueltas de la aguja de los minutos habr\'an sido tres. Y
as\'\i{} para cada uno de los puntos de encuentro subsiguientes. Lo que es lo
mismo a decir:
\begin{align*}
t_2 = \frac {2 \times 360\,^\circ} {\omega_m - \omega_h} \\
t_3 = \frac {3 \times 360\,^\circ} {\omega_m - \omega_h} \\
t_4 = \frac {4 \times 360\,^\circ} {\omega_m - \omega_h} \\
t_5 = \frac {5 \times 360\,^\circ} {\omega_m - \omega_h} \\
t_6 = \frac {6 \times 360\,^\circ} {\omega_m - \omega_h} \\
t_7 = \frac {7 \times 360\,^\circ} {\omega_m - \omega_h} \\
t_8 = \frac {8 \times 360\,^\circ} {\omega_m - \omega_h} \\
t_9 = \frac {9 \times 360\,^\circ} {\omega_m - \omega_h} \\
t_{10} = \frac {10 \times 360\,^\circ} {\omega_m - \omega_h} \\
t_{11} = \frac {11 \times 360\,^\circ} {\omega_m - \omega_h} \\
\end{align*}

Con lo que podemos construir la siguiente tabla:

\begin{center}
\begin{tabular}{|c|l|}
\hline
Encuentro & Hora \\
\hline
$t_1$ & $1\,\text{hora}\,5\,\text{min}\,27.\overline{27}\,\text{seg}$ \\
$t_2$ & $2\,\text{horas}\,10\,\text{min}\,54.\overline{54}\,\text{seg}$ \\
$t_3$ & $3\,\text{horas}\,16\,\text{min}\,21.\overline{81}\,\text{seg}$ \\
$t_4$ & $4\,\text{horas}\,22\,\text{min}\,49.\overline{09}\,\text{seg}$ \\
$t_5$ & $5\,\text{horas}\,27\,\text{min}\,16.\overline{36}\,\text{seg}$ \\
$t_6$ & $6\,\text{horas}\,32\,\text{min}\,43.\overline{63}\,\text{seg}$ \\
$t_7$ & $7\,\text{horas}\,38\,\text{min}\,10.\overline{90}\,\text{seg}$ \\
$t_8$ & $8\,\text{horas}\,43\,\text{min}\,38.\overline{18}\,\text{seg}$ \\
$t_9$ & $9\,\text{horas}\,49\,\text{min}\,5.\overline{45}\,\text{seg}$ \\
$t_{10}$ & $10\,\text{horas}\,54\,\text{min}\,32.\overline{72}\,\text{seg}$ \\
$t_{11}$ & $12\,\text{horas}$ \\
\hline
\end{tabular}
\end{center}

A partir de $t_{11}$ los puntos de encuentro comienzan a repetirse con los ya
calculados (por ejemplo, $t_{12} = t_{1}$).

\begin{description}
\item[Nota:]
N\'otese que, al contrario de lo que se pod\'ia llegar a creer en un primer
    momento, los puntos de encuentro a lo largo de una vuelta entera de la
    aguja de las horas son once y no doce. Esto sucede porque para el
    intervalo que va desde pasadas las \mbox{11 horas} hasta la 1 s\'olo
    existe el encuentro de las \mbox{12 horas} (el primer encuentro despu\'es
    de las \mbox{12 horas} es ya algo m\'as de pasados \mbox{5 minutos} de la
    1). O sea que para ese intervalo de 2 horas s\'olo hay un \'unico
    encuentro y por lo tanto, la cantidad de encuentros totales es uno menos
    que la cantidad de horas en la esfera del reloj.
\end{description}

\end{description}
\end{document}
